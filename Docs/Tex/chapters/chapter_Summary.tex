\chapter*{Összefoglalás}
\addcontentsline{toc}{section}{Összefoglalás}

\section{Elért eredmények}
A szakdolgozat írása során rengeteg új tudásra tettem szert. Megismerkedtem a CMAKE build rendszerrel és a C++ libraryk-el. Mélyebb belátást szereztem a C++ fordítók preprocesszorának és linkerének működésébe. A C++ programozási kézségeim bővültek a lambda függvények, templatek, fájlkezelés, memóriakezelés, debugging és más egyéb std könyvtár által biztosított funkciók ismerettségével. Implementáltam egy egyszerű grafikus motort az OpenGL grafikus API-n keresztül és shader kódokat írtam a GLSL shader nyelven. Megtanultam az ImGui felhasználói felület használtatát és belátást szereztem a képfeldolgozás világába az SSIM és a CSD algoritmusok implementálásan keresztül.

Sikerült egy C++ alkalmazást készítenem ami működik Windows és Linux rendszereken is, alacsony a gépigénye, kevés erőforrást használ, a felhasználói felület pedig átlátható és reszponzív. Az alkalmazás indítása után pedig szinte rögtön futtatható shader kód.

Az alkalmazás minden része nyílt forráskódú és akárki számára elérhető, így bárki tovább fejlesztheti aki akarja vagy felhasználhatja saját céljaira.


\section{Továbbfejlesztési lehetőségek}
Utólag visszatekintve rengeteg mindent máshogy csinálnék és nem mindent sikerült a legszebben megvalósítanom. Ezért az alkalmazás egyes részeit célszerű lenne újratervezni és írni.

Az elvárt funkciókat mind sikeresen implementáltam, de rengeteg új funkcióval lehetne felruházni az alkalmazást. Köztük modellek betöltése, VR támogatás, három dimenziós kép összehasonlítás, számítási shaderek támogatása, Vulkan API implementálása, feladatmegosztó webalkalmazás vagy akár egy teljes interaktív oktató jellegű játék steamworks integrációval.
