\section{Widgetek}

A widgetek, másnéven window objectek, olyan objektumok amik a felhasználói felületet reprezentálják. A widgetek felelősek az adatok megjelenítéséért, a felhasználói interakció kezeléséért.

\subsection{Widget interface}
A widgeteket egy interface-ként valósítottam meg. C++-ban az interface egy olyan osztály, ami csak virtual függvényekből áll.

\begin{minted}
[bgcolor=lbcolor,showspaces=false,breaksymbolleft=,breaklines=true]
{cpp}
class Widget {
  public:
    virtual void OnUpdate(float deltaTime) {};
    virtual void OnRender() {};
    virtual void OnImGuiRender() {};
    virtual void RenderWidget() {};
};
\end{minted}

Az alkalmazásban minden widget ezt az interfacet implementálja. Ez segít abban, hogy a widgeteket generikusan tudjuk majd pélányosítani upcastolás segítségével. Az \mintinline{cpp}{OnUpdate()} függvényben történik a változók képkockánkénti frissítése, az \mintinline{cpp}{OnRender()}-ben pedig az OpenGL render hívások, majd az \mintinline{cpp}{OnImGuiRender()}-ben az ImGui render hívások, ez a függvény mindig egy ImGui ablak context-ba van ágyazva.

\newpage

\begin{minted}
[bgcolor=lbcolor,showspaces=false,breaksymbolleft=,breaklines=true]
{cpp}
void TaskWidget::RenderWidget() {
  OnUpdate(0.0f);
  OnRender();
  ImGui::Begin("windowName");
  OnImGuiRender();
  ImGui::End();
}
\end{minted}

\subsection{Widget broker}

A widgetek példányosításáért egy \mintinline{cpp}{WidgetBroker} osztály felel. Az osztály \mintinline{cpp}{MakeWidget()} metódusa template paraméterként várja, hogy milyen típusú Widgetet készítsen, függvény paraméterként pedig egy \mintinline{cpp}{WidgetType} enum értéket kap, ami a widget azonosítója lesz, ezután pedig a példányosítani kívánt widget konstruktorának a paramétereit adhajtuk át. Végül eltároljuk egy map-ben a widgetet és visszatérünk egy pointerral ami egy unique smart pointer-ra mutat ami maga a widget.

\begin{minted}
[bgcolor=lbcolor,showspaces=false,breaksymbolleft=,breaklines=true]
{cpp}
template <typename WidgetT, typename... Arguments>
Widget* MakeWidget(WidgetType id, Arguments&&... ArgumentValues) {
  assert(m_Widgets.count(id) != 1);
  m_Widgets[id] = std::make_unique<WidgetT>(
    std::forward<Arguments>(ArgumentValues)...);
  return m_Widgets[id].get();
}
\end{minted}

Az alkalmazás különböző pontjain tudnunk kell az egyes widgetek-et manipulálni, például ha valamilyen funkcionalist egy gyorsbillentyűhöz akarunk kötni akkor a GLFW bevitel kezelő callback függvényeiben meg kell tudnunk hívni egy widget publikus metódusát. Ezt a \mintinline{cpp}{GetWidget()} függvénnyel tehetjük meg, ami azonosító alapján megkeresi a map-ben a használni kívánt widget-et és visszaad egy arra mutató pointert, amit upcastolunk a generikusként megkapott widget típusra.

\begin{minted}
[bgcolor=lbcolor,showspaces=false,breaksymbolleft=,breaklines=true]
{cpp}
template <typename WidgetT>
WidgetT* GetWidget(WidgetType id) {
      return (WidgetT*)m_Widgets[id].get();
}
\end{minted}

Példa a használatára.

\begin{minted}
[bgcolor=lbcolor,showspaces=false,breaksymbolleft=,breaklines=true]
{cpp}
context->widgetBroker
  .GetWidget<TextEditorWidget>("TextEditor")->Save();
\end{minted}

Az alkalmazásban szereplő widgetek:
\begin{itemize}
    \item Text editor widget: shader forrásfájlok szerkesztése
    \item Scene editor widget: scene elementek létrehozása, törlése, clear color beállítása és a scene element adatainak beállítása
    \item Console widget: oda-vissza kommunikáció az alkalmazás és a felhasználó között
    \item File browser widget: fájl böngésző amin keresztül fájlok elérési útjait tudjuk megadni
    \item Task widget: cél és a rajzolt kép közlése a felhasználóval, szintén ezen a widgeten keresztül tudunk feladatot beállítani és a képeket összehasonlítani.
\end{itemize}