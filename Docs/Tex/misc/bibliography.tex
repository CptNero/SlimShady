%% Az itrodalomjegyzek keszitheto a BibTeX segedprogrammal:
%\bibliography{diploma}
%\bibliographystyle{plain}

%VAGY "kézzel" a következõ módon:

\begin{thebibliography}{9}
%10-nél kevesebb hivatkozás esetén

%\begin{thebibliography}{99}
% 10-nél több hivatkozás esetén

\addcontentsline{toc}{section}{Irodalomjegyzék}

%Elso szerzok vezetekneve alapjan ábécérendben rendezve.


%folyóirat cikk: szerzok(k), a folyóirat neve kiemelve,
%az evfolyam felkoveren, zarojelben az evszam, vegul az oldalszamok es pont.
\bibitem{OpenGL}
A hivatalos OpenGL weboldal.
\newline \url{https://www.khronos.org/opengl/}

\bibitem{CSQD}
H. Sadeghi and A. Raie, "Approximated Chi-square distance for histogram matching in facial image analysis: Face and expression recognition," 2017 10th Iranian Conference on Machine Vision and Image Processing (MVIP), Isfahan, Iran, 2017, pp. 188-191, doi: 10.1109/IranianMVIP.2017.8342346.
\newline \url{https://ieeexplore.ieee.org/document/8342346}

\bibitem{SSIM}
Z. Wang, E. P. Simoncelli and A. C. Bovik, "Multiscale structural similarity for image quality assessment," The Thrity-Seventh Asilomar Conference on Signals, Systems \& Computers, 2003, Pacific Grove, CA, USA, 2003, pp. 1398-1402 Vol.2, doi: 10.1109/ACSSC.2003.1292216.
\newline \url{https://ieeexplore.ieee.org/document/1292216}

\bibitem{OpenGLGraphicsPipeline}
Az OpenGL grafikus csővezeték \newline
\url{https://www.researchgate.net/profile/Christoph-Guetter/publication/235696712/figure/fig1/AS:299742132228097@1448475501091/The-graphics-pipeline-in-OpenGL-consists-of-these-5-steps-in-the-new-generation-of-cards.png}

\bibitem{RGB}
Az RGBA komponens \newline
\url{https://olcovers2.blob.core.windows.net/coverswp/2016/11/rgba-pixel-model.png}

\bibitem{vertexlayout}
Vertex attribútumok elrendezése a memóriában
\newline \url{https://learnopengl.com/img/getting-started/vertex_attribute_pointer_interleaved_textures.png}

\bibitem{Joey}
Joey De Vries.
Learn OpenGL tutorials \newline
\url{https://learnopengl.com}

\end{thebibliography}