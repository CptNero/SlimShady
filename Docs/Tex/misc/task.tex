%A \chapter* parancs nem ad a fejezetnek sorszámot
\chapter*{Feladatkiírás}
%A tartalomjegyzékben mégis szerepeltetni kell, mint szakasz(section) szerepeljen:
\addcontentsline{toc}{section}{Feladatkiírás}

\section*{OpenGL/ES Shader programozó játék}

Az alkalmazás célja, hogy a segítse a kezdő OpenGL(ES) programozókat és
különféle feladatok során gyakoroltassa a shader programok írását, konfigurálását.

A programban megadott rajzolási feladatokat kell a felhasználónak megoldania.

Az alkalmazás a felhasználó által megadott shader és uniform adatok alapján
OpenGL (ES) felhasználásával rajzol ki egy képet. Ezt a képet az éppen aktuális
feladat által megkövetelt eredmény képhez hasonlítja a program. Amennyiben
a felhasználó által "készített" kép megegyezik a feladatban elvárttal, akkor jöhet
a következő feladat.

A felhasználói felület interaktívnak kell legyen, azaz ha pl.: a shader kód
módosul, akkor az az által kirajzolandó képnek meg kell jelennie
(akár automatikusan akár egy gomb lenyomásával) anélkül, hogy magát az alkalmazást
újrafordítanánk.

A felhasználó számára biztosítani kell a következő lehetőségeket. A háttér szín megadását, (clear color), a vertex és fragment shaderek módosíthatóságát, vertex input attribútumok, megadását, tetszőleges uniformok hozzáadását (név, típus, érték), tetszőleges samplerek/képek megadását (név, típus, képfájl).

Adjunk lehetőséget a "sandbox" módra is. Lehetőség szerint a feladatok listájának bővíthetőnek
kell lennie. Az alkalmazás újrafordítása nélkül lehessen további feladatokat hozzáadni.

A megvalósításnak modern OpenGL (ES)-ben kell történnie, célszerűen Windows és Linux platformon
is működnie kell és nem lehet Web-es megvalósítású (nem WebGL).