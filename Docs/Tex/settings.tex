\documentclass[a4paper,12pt]{report}

\usepackage[utf8]{inputenc}
\usepackage{t1enc}
\usepackage[magyar]{babel}
% A formai kovetelmenyekben megkövetelt Times betûtípus hasznalata:
\usepackage{times}

%Az AMS csomagjai
\usepackage{amsmath}
\usepackage{amssymb}
\usepackage{amsthm}

%A fejléc láblécek kialakításához:
\usepackage{fancyhdr}

%Természetesen további csomagok is használhatók,
%például ábrák beillesztéséhez a graphix és a psfrag,
%ha nincs rájuk szükség természetesen kihagyhatók.
\usepackage{graphicx}
\usepackage[activate={true,nocompatibility},final,tracking=true,kerning=true,spacing=true,factor=1100,stretch=10,shrink=10]{microtype}
\usepackage{psfrag}
\usepackage{setspace}
\usepackage{qtree}
\usepackage[edges]{forest}
\usepackage{geometry}
\usepackage{hyperref}
\hypersetup{
    colorlinks=true,
    linkcolor=blue,
    filecolor=magenta,
    urlcolor=cyan,
    citecolor=blue
}
\urlstyle{same}

\usepackage{minted}

%tetelszerû környezetek definiálhatók, ezek most fejezetenkent egyutt szamozodnak, pl.
\newtheorem{tet}{tetel}[chapter]
\newtheorem{defi}[tet]{Definíció}
\newtheorem{lemma}[tet]{Lemma}
\newtheorem{áll}[tet]{Állítás}
\newtheorem{köv}[tet]{Következmény}

%Ha a megjegyzések és a példak szövegét nem akarjuk dõlten szedni, akkor
%az alábbi parancs után kell õket definiální:
\theoremstyle{definition}
\newtheorem{megj}[tet]{Megjegyzés}
\newtheorem{pld}[tet]{Példa}

%Margók:
\hoffset -1in
\voffset -1.5in
\topmargin 1in
\oddsidemargin 35mm
\textwidth 150mm
\headheight 10mm
\headsep 5mm
\textheight 220mm
\linespread{1,5}

\definecolor{lbcolor}{rgb}{0.95,0.95,0.95}